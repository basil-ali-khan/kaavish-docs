\section{Introduction}
Modern enterprise systems generate vast amounts of log data, challenging both manual and traditional approaches for effective anomaly detection, root-cause analysis, and self-healing. Manual rule-based monitoring has long been insufficient for the scale and complexity of cloud-native systems, particularly as log volumes expand exponentially and overwhelm human operators~\cite{guntupalli2025aidriven, cabello2025aiops}
. The rise of AI for IT Operations (AIOps) marked a shift toward automating log understanding, anomaly detection, and debugging using machine learning and deep learning~\cite{ji2025llmid, guntupalli2025aidriven, cabello2025aiops}
.

\section{Current State-of-the-Art}

Recent advances in log anomaly detection have focused on leveraging deep sequential models, transformer architectures, and increasingly, large language models (LLMs). Early approaches such as DeepLog~\cite{du2017deeplog} employed LSTM-based architectures to model log sequences and detect deviations from normal patterns. Transformer-based successors like LogBERT~\cite{guo2021logbert} improved context encoding and demonstrated stronger generalization.

Traditional and early semi-supervised approaches, however, suffered from practical challenges including intensive feature engineering, limited scalability, brittleness under log evolution (concept drift), and a heavy reliance on labeled data for tuning~\cite{guntupalli2025aidriven, albert2025systemlogs, cabello2025aiops}. These limitations hindered reliable deployment in fast-changing, large-scale production systems.

Cutting-edge frameworks now employ both transformer-based LLMs and reinforcement learning for superior performance and adaptability. LogLLaMA~\cite{yang2025logllama}, for example, fine-tunes LLaMA2 on normal log sequences and incorporates RL to maximize anomaly detection accuracy, achieving state-of-the-art F1 scores across HDFS, BGL, and Thunderbird benchmarks~\cite{yang2025logllama}. Similarly, LogGPT~\cite{han2023loggpt} adapts GPT-2 for log prediction and uses RL-style optimization for extracting robust anomaly signals.

Beyond these models, Siamese-based approaches such as SiaLog~\cite{hashemi2022sialog} enhance robustness to log evolution and class imbalance. Comparative evaluations consistently show that SiaLog, transformer methods, and LLM-based techniques outperform traditional baselines such as PCA, OCSVM, Isolation Forest, and invariant mining—especially on noisy or evolving logs~\cite{hashemi2022sialog, yang2025logllama, han2023loggpt}.

LLMs introduce new opportunities by directly interpreting raw, unstructured log messages using their semantic reasoning capabilities~\cite{barenji2025agenticai, cabello2025aiops}. However, their generality can lead to hallucinations or inaccuracies in highly specialized operational domains. This motivated the integration of Retrieval-Augmented Generation (RAG), which supplements LLMs with contextual, domain-specific knowledge retrieved from external log corpora, documentation, or vector databases~\cite{cabello2025aiops}. RAG-enabled frameworks (e.g., retrieval-guided log analysis systems) improve accuracy, interpretability, and stability by grounding LLM outputs in relevant operational context~\cite{cabello2025aiops}.

\section{Similar Work in the Domain}

Beyond detection, researchers have increasingly explored agentic AI for automated remediation and root-cause analysis in enterprise environments. Guguloth~\cite{guguloth2025selfhealing} surveys AI-driven self-healing frameworks that monitor logs, correlate multivariate evidence, and orchestrate autonomous repair actions. Sivakumar~\cite{sivakumar2024agentic} describes agentic AIOps architectures using predictive analytics, planning components, and multi-agent coordination to reduce human workload and accelerate recovery.

Traditional log analytics pipelines, such as ELK-based platforms, aggregate and index logs but rely on human interpretation or simple rules~\cite{kleindienstlogging}. Newer frameworks incorporate LLMs and retrieval mechanisms to provide contextual explanations, generate hypotheses, and guide operational workflows~\cite{cabello2025aiops}. Some recent systems further integrate tool-use, memory, and goal-directed planning (key elements of agentic AI~\cite{barenji2025agenticai}) enabling agents to triage incidents, gather additional evidence, and drive remediation workflows autonomously.

The Intelligent Debugger (LLM-ID) paradigm is a notable example: it combines fine-tuned LLMs for multi-stage semantic inference with an RL-based policy planner to recommend or execute remediation actions~\cite{ji2025llmid, barenji2025agenticai}. Such systems demonstrate the transition from simple anomaly detection toward autonomous debugging pipelines capable of structured reasoning and adaptive decision-making in cloud environments.

\section{Differences and Contributions of This Work}

Whereas prior work primarily focuses on detection—DeepLog~\cite{du2017deeplog}, LogBERT~\cite{guo2021logbert}, LogGPT~\cite{han2023loggpt}, LogLLaMA~\cite{yang2025logllama}, SiaLog~\cite{hashemi2022sialog}—and some examine system-level remediation~\cite{guguloth2025selfhealing, sivakumar2024agentic}, most do not unify (a) context-aware anomaly explanation, (b) nuanced, multi-stage root cause reasoning, and (c) autonomous, hierarchical remediation within a cohesive agentic framework.

Our work addresses this gap by integrating state-of-the-art transformer/LLM methods with an autonomous multi-agent orchestration framework capable of structured diagnosis, evidence gathering, planning, and adaptive follow-up actions (e.g., escalation, rollback, parameter tuning, resource management). By incorporating retrieval augmentation and feedback-driven online learning, our architecture enhances robustness to log evolution, mitigates LLM hallucination risk, and maintains performance on unseen failure or attack patterns~\cite{cabello2025aiops}. We demonstrate the end-to-end chain—from detection to reasoning to remediation—in real-time enterprise log settings and validate performance on open benchmarks~\cite{yang2025logllama, hashemi2022sialog}.

\section{Conclusion}
State-of-the-art research highlights both the strengths and remaining limitations of current AI-driven log anomaly detection and remediation. While LLMs, RAG, and reinforcement learning have advanced detection and interpretability, fully autonomous, agentic frameworks capable of end-to-end reasoning and enterprise-grade remediation remain an open frontier. Our work moves toward closing this gap by unifying LLM-based detection, retrieval-enhanced reasoning, and agentic remediation into a cohesive and adaptive solution.
