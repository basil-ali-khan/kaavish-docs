\section{Introduction}
Modern enterprise systems generate vast amounts of log data, challenging both manual and traditional approaches for effective anomaly detection, root-cause analysis, and self-healing. Machine learning—especially deep learning—has rapidly become state-of-the-art in automating these tasks.

\section{Current State-of-the-Art}

Recent advances in log anomaly detection have focused on leveraging deep sequential models and large language models (LLMs). Early approaches, such as DeepLog~\cite{du2017deeplog}, utilized LSTM-based architectures to model log message sequences and detect deviations from normal patterns. Successors like LogBERT applied the transformer architecture to better encode log event context~\cite{guo2021logbert}.

Cutting-edge frameworks now employ both transformer-based LLMs and reinforcement learning for superior performance and adaptability. Notably, LogLLaMA~\cite{yang2025logllama} fine-tunes LLaMA2 on normal log sequences, then applies reinforcement learning to maximize anomaly detection accuracy. It has demonstrated state-of-the-art F1 scores across benchmark datasets (HDFS, BGL, Thunderbird), outperforming LSTM-based, BERT-based, and classical baselines~\cite{yang2025logllama}. Meanwhile, LogGPT~\cite{han2023loggpt} directly adapts GPT-2 for log prediction and uses RL-style finetuning for robust anomaly signal extraction.

SiaLog~\cite{hashemi2022sialog} advances anomaly detection by applying a Siamese network atop RNNs, improving robustness to log evolution and class imbalance. Comparative experiments show that SiaLog, transformer approaches, and LLM-based solutions consistently outperform older methods like PCA, OCSVM, and invariant mining, particularly on evolving or noisy logs~\cite{hashemi2022sialog, yang2025logllama, han2023loggpt}.

\section{Similar Work in the Domain}

Researchers have recently addressed not just anomaly detection, but agentic AI for autonomous remediation and root cause analysis in enterprise environments. Guguloth~\cite{guguloth2025selfhealing} surveys AI-powered self-healing frameworks that monitor logs, reason over multivariate evidence, and orchestrate self-repair actions. Sivakumar~\cite{sivakumar2024agentic} summarizes agentic AIOps and multi-agent systems that use predictive analytics and machine learning for IT autonomy, driving down mean time-to-repair and human workload.

Traditional log analytics pipelines, such as ELK-based platforms, perform log aggregation and indexing but require human or simple rule-based intervention~\cite{kleindienstlogging}. In contrast, recent frameworks support self-adaptive agents that monitor, diagnose, and execute remediation with little or no human input, integrating AI-driven knowledge with infrastructure automation~\cite{guguloth2025selfhealing, sivakumar2024agentic}.

\section{Differences and Contributions of This Work}

Whereas prior work largely specializes in detection using models—DeepLog~\cite{du2017deeplog}, LogBERT~\cite{guo2021logbert}, LogGPT~\cite{han2023loggpt}, LogLLaMA~\cite{yang2025logllama}, SiaLog~\cite{hashemi2022sialog}—and some focus on system-level automated remediation~\cite{guguloth2025selfhealing, sivakumar2024agentic}, most do not close the loop between context-aware anomaly explanation, nuanced root cause reasoning, and hierarchical, intelligent remediation in a fully autonomous, agentic framework.

Our work addresses this gap by (a) leveraging state-of-the-art transformer/LLM methods proven effective in both theoretical and empirical studies, (b) tightly coupling these with an autonomous multi-agent orchestration framework capable of both structured diagnosis and adaptive follow-up actions (escalation, rollback, parameter tuning, resource management), and (c) demonstrating this chain in a real-time, evolving enterprise log setting, with evaluation on open datasets. Our architecture further incorporates feedback and online learning, improving robustness to log evolution and unseen attack/failure types, as validated in recent benchmarks~\cite{yang2025logllama, hashemi2022sialog}.

\section{Conclusion}
State-of-the-art research demonstrates the strengths and limits of current AI-driven log anomaly detection and remediation. However, fully autonomic, reasoning-capable agentic frameworks for root cause analysis and enterprise-grade remediation remain open frontiers; our work advances the field by unifying LLM-based detection with agentic remediation in a cohesive, adaptive solution.
